
\section{Methodology and Development Process}

\subsection{Methodology used}

%Describe the methodology\cite{despa2014comparative} followed (example: iterative development with fortnightly sprints and weekly follow-up meetings) and resources used (example: GitHub \cite{github}, etc.).	
The project followed an iterative development approach, with weekly meetings to discuss progress, analyze findings, evaluate the results, and plan next steps.  Thanks to this process, as the project developed, we were able to get feedback and make the required changes. 

The key resources used during the project included GitHub for report collaboration, PyTorch for neural network training, and TransE embeddings for the KG4Vis model. The team also relied on the original papers for VizML and KG4Vis to guide the implementation and analysis.	

\subsection{Stakeholders, roles and responsibilities}

%Identify the project team, stakeholders and other investors with whom there was interaction; in the case of group work, clarify the roles and responsibilities of each member of the group.

There were several participants in this project, and each had different roles and responsibilities contributing to the project's success:

\begin{itemize}
\item \textbf{Project Coordinator and Tutor} - Professor at Faculdade de Engenharia da Universidade do Porto Alexandre Miguel Barbosa Valle de Carvalho, responsible for guiding the project, providing feedback, and evaluating the final results.
\item \textbf{Team Members and Distribution of Work} 

- João Miguel Peixoto Lamas (up202208948): Initial research, information gathering, model testing, and analysis of results (25\%).

- Pedro Afonso Nunes Fernandes (up202207987): Initial research, information gathering, and subsequent refinement of the collected information to build the report (25\%).

- Tiago de Pinho Bastos de Oliveira Pinheiro (up202207890): Initial research, information gathering, and subsequent refinement of the collected information to build the report (25\%).

- Tiago Grilo Ribeiro Rocha (up202206232): Initial research, information gathering, model testing, and analysis of results (25\%).
\end{itemize}
\vspace{7cm}    

\subsection{Activities developed}

\begin{table}[h!]
\centering
\caption{Activities developed during the project timeline.}
\begin{tabular}{|l|p{10cm}|}
\hline
\textbf{Timeline} & \textbf{Activities} \\
\hline
13/02/25 & Initial discussion of the VizML paper and verification of the dataset. \\
\hline
20/02/25 & Analysis of the VizML and KG4Vis papers, making explanatory diagrams for each one, dataset characterization, and software installation. \\
\hline
27/02/25 & Exploration of neural network training and embedding learning, start running the dataset. \\
\hline
06/03/25 & Deep dive into scalability challenges, computational costs, and model training preparation. \\
\hline
13/03/25 & Researched graph storage methods, explored training tech, calculated computational costs, corrected prior errors, detailed VizML-KG4Vis pipeline, added references, prepared dataset scripts, and initiated report development. \\
\hline
20/03/25 & Dove deeper into the code, analyzed neural network topology and deep learning model, explored Torch variables and data loaders, implemented simple classifiers, documented dataset file types and execution, and reviewed the report structure. \\
\hline
03/04/25 & Filled initial report sections, refined index, restructured to prioritize "Concepts," added comparative analysis of papers/code (performance/scalability post-execution), included metrics (accuracy, precision), and pre/post-execution comparisons. Concluded with work distribution. \\
\hline
08/04/25 & Restructured report to discuss approach before dataset, reordered sections, added centered image captions. Explored applying functions with AI, documented issues with functions, noting attempted fixes. \\
\hline
24/04/25 & After multiple attempts to adjust parameters and test VizML, concluded it wasn't yielding meaningful results and shifted focus to running the original KG4Vis code as implemented by the authors. \\
\hline
07/05/25 & Filled missing report topics and tested KG4Vis code for different dimensions, changing some testing parameters (inferior batch size, hidden dimension). \\
\hline
15/05/25 & Finalized testing and documentation, prepared for project submission. \\ 
\hline
\end{tabular}
\end{table}



%Describe the activities carried out over time (including relevant events such as presentations, meetings with clients, etc.) and their respective deliverables, typically using a Gantt chart and a summary description of each activity/deliverable. 



